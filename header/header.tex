%BuG-Fix
%Package pdf Error: Driver file ................ not found
%If you have a luatex driver fail uncomment these lines
%\RequirePackage{luatex85}
%\def\pgfsysdriver{pgfsys-pdftex.def}

% Genereller Header
\documentclass[11pt,twoside,a4paper,fleqn]{article}
% Dateiencoding
\usepackage[utf8]{inputenc}
\usepackage[T1]{fontenc}	%ä,ü...
% Seitenränder
\usepackage[left=1cm,right=1cm,top=0.5cm,bottom=0.2cm,includeheadfoot]{geometry}
\setlength{\headsep}{5pt} 
% Sprachpaket 
\usepackage[english, ngerman]{babel} % Silbentrennung und Rechtschreibung Englisch und Deutsch

%%%%%%%%%%%%%%%%%%%%%%%
%% Wichtige Packages %%
%%%%%%%%%%%%%%%%%%%%%%%
\usepackage{amsmath}                % Allgemeine Matheumgebungen									
%\usepackage{amssymb}                % Fonts: msam,msbm, eufm & Mathesymbole, Mengen (lädt automatisch amsfonts)									
\usepackage{array}                  % \newcolumntype, \firsthline, ,\lasthline, m{width}, b{width}									
\usepackage{caption}                % Bildunterschriften									
\usepackage{enumitem}               % basic environments: enumerate, itemize, description									
\usepackage{fancybox}               % \fbox: \shad­ow­box, \dou­ble­box, \oval­box, \Oval­box									
\usepackage{fancyhdr}               % Seiten schöner gestalten, insbesondere Kopf- und Fußzeile									
\usepackage{floatflt}               % Textumflossene Abbildungen \begin{floatingfigure}[r]{Breite} : r rechts, l links, p links auf geraden Seiten und rechts auf ungeraden Seiten								
\usepackage{graphicx}               % \includegraphics[keyvals]{imagefile}, [draft]graphicx zeigt nur Namen und Rahmen an, [final] hebt diese option auf => Bild wird angezeigt    									
\usepackage{hyperref}               % Erstellt Verweise innerhalb und nach außerhalb eines PDF Dokumentes.									
\usepackage{lastpage}               % Bspw. : Page 1 of 3 => \thepage\ of \pageref{LastPage}									
\usepackage{listings}               % Erlaubt es Programmcode in der gewünschten Sprache zu hinterlegen (C++, Matlab,..). Definition der Sprache mit \lstset{language=name}..									
\usepackage{longtable}              % Longtable erlaubt es Tabellen zu erstellen die bei der nächsten Seite weiterlaufen. (Bricht automatisch um)									
%\usepackage{mathabx}                % Mathesymbole									
%\usepackage{mathrsfs}               % \mathscr (Benötigt für Fourierreihen-Symbol)									
%\usepackage{mathtools}              % Extension package to amsmath									
\usepackage{multicol}               % multicols-Umgebung \begin{multicols}{3} erzeugt Abschnitt mit 3 Spalten									
\usepackage{multirow}               % Tabelle: ermöglicht es Felder mehrerer Zeilen in einem zusammenzufassen									
\usepackage{pdflscape}              % adds PDF support to the environment 'landscape'									
\usepackage{pxfonts}                % Symbole, griechisches Alphabet, Integrale...									
\usepackage{rotating}               % sideways, turn{degree}, rotate{degree}, sidewaysfigure, sidewaystable Umgebung									
\usepackage{subcaption}             % Bildunterschriften für Subfigures									
\usepackage{tabularx}               % tabularx-Umgebung: Hat feste Gesamtbreite, \begin{tabularx}{\textwidth}{c c c c c} X: Spalte mit variabler Breite, l, c, r, p{breite}, m{breite}									
%\usepackage{textcomp}               % text symbols: baht, bullet, copyright, musical-note, onequarter, section, yen																	
\usepackage{titlesec}               % Überschriften zu Textabstände
\usepackage{trfsigns}               % Transformationszeichen \laplace, \Laplace..									
\usepackage{trsym}                  % Weitere Laplace Zeichen erlaubt auch vertikale Transformationszeichen									
\usepackage{verbatim}               % verbatim, verbatim*, comment Umgebung									
\usepackage{wrapfig}                % Textumflossene Bilder und Tabellen, \begin{wrapfigure}[Zeilen]{Position}[Ueberhang]{Breite}									
%\usepackage{xcolor}                 % \pagecolor{color}, \textcolor{color}{text}, \colorbox{color}{text}, \fcolorbox{border-color}{fill-color}{text}
\usepackage[table]{xcolor}			% Tabellenfarbe	\rowcolors vor \begin{tabularx}...							
\usepackage{titlesec}
% Zum Bilder einfach in Tabellen einfügen (valign=t)
\usepackage[export]{adjustbox}
\usepackage{tikz}                   % Tikz Umgebung zur Grafikerzeugung	
\usetikzlibrary{fit}
\usetikzlibrary{arrows.meta}

%%%%%%%%%%%%%%%%%%%%
% Generelle Makros %
%%%%%%%%%%%%%%%%%%%%
\newcommand{\skript}[1]{$_{\textcolor{red}{\mbox{\small{Skript S.#1}}}}$}
\newcommand{\verweis}[2]{\small{(siehe auch \ref{#1}, #2 (S. \pageref{#1}))}}
\newcommand{\verweiskurz}[1]{(\small{siehe \ref{#1}\normalsize)}}
\newcommand{\subsubadd}[1]{\textcolor{black}{\mbox{#1}}}
\newcommand{\formelbuch}[1]{$_{\textcolor{red}{\mbox{\small{S#1}}}}$}

\newcommand{\kuchling}[1]{$_{\textcolor{red}{\mbox{\small{Kuchling #1}}}}$}
\newcommand{\stoecker}[1]{$_{\textcolor{grey}{\mbox{\small{Stöcker #1}}}}$}
\newcommand{\sachs}[1]{$_{\textcolor{blue}{\mbox{\small{Sachs S. #1}}}}$}
\newcommand{\hartl}[1]{$_{\textcolor{green}{\mbox{\small{Hartl S. #1}}}}$}

\newcommand{\schaum}[1]{\tiny Schaum S. #1}

\newcommand{\skriptsection}[2]{\section{#1 {\tiny Skript S. #2}}}
\newcommand{\skriptsubsection}[2]{\subsection{#1 {\tiny Skript S. #2}}}
\newcommand{\skriptsubsubsection}[2]{\subsubsection{#1 {\tiny Skript S. #2}}}

\newcommand{\matlab}[1]{\footnotesize{(Matlab: \texttt{#1})}\normalsize{}}

\newcommand\tabbild[2][]{%
	\raisebox{0pt}[\dimexpr\totalheight+\dp\strutbox\relax][\dp\strutbox]{%
		\includegraphics[#1]{#2}%
	}%
}

\newcolumntype{P}[1]{>{\raggedright\arraybackslash}p{#1}} %Tabelle linksausgerichtet
\newcolumntype{L}[1]{>{\raggedleft\arraybackslash}p{#1}} %Tabelle rechtsausgerichtet
\newcolumntype{C}[1]{>{\centering\arraybackslash}p{#1}}

%%%%%%%%%%%%
% Listings %
%%%%%%%%%%%%

\makeatletter
\newcommand\processAsterisk{%
	\ifnum\lst@mode=\lst@Pmode\relax%
	\textcolor{black}{*}%
	\else
	*%
	\fi
}
\makeatother

\lstset{
	language=C++,
	backgroundcolor=\color{timberwolf},
	frame=single,
	keywordstyle=\color{blue},
	breaklines=true,
	escapeinside={*@}{@*},
	rulecolor=\color{black},
	title=\lstname,
	morekeywords={*, const},
	tabsize = 2,
	literate={*}{\processAsterisk}1
}
\def\listings{./listings}	% Dateipfad für Listings


%%%%%%%%%%%%%%%
% Achtung-Box %
%%%%%%%%%%%%%%%
\newenvironment{achtung}[1][Achtung]{
	\rule{\textwidth}{1pt}\\
	\textbf{#1}:
}{
	\\\rule[1ex]{\textwidth}{1pt}
}

%%%%%%%%%%%%%%%
% Hinweis-Box %
%%%%%%%%%%%%%%%
\newenvironment{hinweis}[1][Hinweis]{
	\rule{\textwidth}{1pt}\\
	\textbf{#1}:
}{
	\\\rule[1ex]{\textwidth}{1pt}
}

%%%%%%%%%%
% Farben %
%%%%%%%%%%
\definecolor{black}{rgb}{0,0,0}
\definecolor{red}{rgb}{1,0,0}
\definecolor{white}{rgb}{1,1,1}
\definecolor{grey}{rgb}{0.8,0.8,0.8}
\definecolor{green}{rgb}{0,.8,0.05}
\definecolor{brown}{rgb}{0.603,0,0}
\definecolor{mymauve}{rgb}{0.58,0,0.82}
\definecolor{timberwolf}{rgb}{0.86, 0.84, 0.82}

%%%%%%%%%%%%%%%%%%%%%%%%%%%%
% Mathematische Operatoren %
%%%%%%%%%%%%%%%%%%%%%%%%%%%%
\DeclareMathOperator{\sinc}{sinc}
\DeclareMathOperator{\sgn}{sgn}
\DeclareMathOperator{\Real}{Re}
\DeclareMathOperator{\Imag}{Im}
%\DeclareMathOperator{\e}{e}
\DeclareMathOperator{\cov}{cov}
\DeclareMathOperator{\PolyGrad}{PolyGrad}

%Grösse Integral anpassen
\def\Int{\mbox{\Large$\displaystyle\int$\normalsize}}
\def\OInt{\mbox{\Large$\displaystyle\oint$\normalsize}}

%Makro für 'd' von Integral- und Differentialgleichungen 
\newcommand*{\diff}{\mathop{}\!\mathrm{d}}

%%%%%%%%%%%%%%%%%%%%%%%%%%%
% Fouriertransformationen %
%%%%%%%%%%%%%%%%%%%%%%%%%%%

% Fouriertransformationen
\unitlength1cm
\newcommand{\FT}
{
	\begin{picture}(1,0.5)
	\put(0.2,0.1){\circle{0.14}}\put(0.27,0.1){\line(1,0){0.5}}\put(0.77,0.1){\circle*{0.14}}
	\end{picture}
}


\newcommand{\IFT}
{
	\begin{picture}(1,0.5)
	\put(0.2,0.1){\circle*{0.14}}\put(0.27,0.1){\line(1,0){0.45}}\put(0.77,0.1){\circle{0.14}}
	\end{picture}
}


%%%%%%%%%%%%%%%%%%%%%%%%%%%%
% Allgemeine Einstellungen %
%%%%%%%%%%%%%%%%%%%%%%%%%%%%

\setitemize{noitemsep,topsep=0pt,parsep=0pt,partopsep=0pt} %kompakte itemize
\setenumerate{noitemsep,topsep=0pt,parsep=0pt,partopsep=0pt} %kompakte enumerate

%Pdf Info
\hypersetup{pdfauthor={\authorname},pdftitle={\titleinfo},colorlinks=false}
\author{\authorname}
\title{\titleinfo}

% Abstände Text zu Übertiteln / Einzug
\titlespacing{\section}{12pt}{1em}{0.5em}
\titlespacing{\subsection}{12pt}{1em}{0.5em}
\titlespacing{\subsubsection}{12pt}{1em}{0.5em}

%%%%%%%%%%%%%%%%%%%%%%%
% Kopf- und Fusszeile %
%%%%%%%%%%%%%%%%%%%%%%%
\pagestyle{fancy}
\fancyhf{}
%Linien oben und unten
\renewcommand{\headrulewidth}{0.5pt} 
\renewcommand{\footrulewidth}{0.5pt}

%Kopfzeile links bzw innen
\fancyhead[L]{\titleinfo{ }\tiny{(\versioninfo)}}
%Kopfzeile mitte
%\fancyhead[C]{}
%Kopfzeile rechts bzw. aussen
\fancyhead[R]{Seite \thepage{ }von \pageref{LastPage}}

%Fusszeile links bzw. innen
\fancyfoot[L]{\footnotesize{\authorname}}
%Fusszeile mitte
%\fancyfoot[C]{\footnotesize{\authoremail}}
%Fusszeile rechts bzw. ausen
\fancyfoot[R]{\footnotesize{\today}}
% Einrücken verhindern versuchen
\setlength{\parindent}{0pt}

%%%%%%%%%%%%%%%%%%%%%%%%%%%%%%%%%%%%%%%
%% Makros & anderer Low-Level bastel %%
%%%%%%%%%%%%%%%%%%%%%%%%%%%%%%%%%%%%%%%
% Zeilenhöhe Tabellen:
\newcommand{\arraystretchOriginal}{1.5}
\renewcommand{\arraystretch}{\arraystretchOriginal}

\makeatletter
%% Makros für den Arraystretch (bei uns meist in Tabellen genutzt, welche Formeln enthalten)
% Default Value
\def\@ArrayStretchDefault{1} % Entspricht der Voreinstellung von Latex

% Setzt einen neuen Wert für den arraystretch
\newcommand{\setArrayStretch}[1]{\renewcommand{\arraystretch}{#1}}

% Setzt den arraystretch zurück auf den default wert
\newcommand{\resetArrayStretch}{\renewcommand{\arraystretch}{\@ArrayStretchDefault}}

% Makro zum setzen des Default arraystretch. Kann nur in der Präambel verwendet werden.
\newcommand{\setDefaultArrayStretch}[1]{%
    \AtBeginDocument{%
        \def\@ArrayStretchDefault{#1}
        \renewcommand{\arraystretch}{#1}
    }
}
\makeatother

%% Achtung Symbol \danger
\newcommand*{\TakeFourierOrnament}[1]{{%
        \fontencoding{U}\fontfamily{futs}\selectfont\char#1}}
\newcommand*{\danger}{\TakeFourierOrnament{66}}