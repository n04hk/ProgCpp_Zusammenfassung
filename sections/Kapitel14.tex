%!TEX root = ProgCPP_ZF.tex

\part{Preprocessor}

\begin{multicols}{2}
\section{Eigenschaften des Preprocessors}
\begin{itemize}
	\item Wird vor der eigentlichen Übersetzung aktiviert
	\item Führt textuelle Manipulationen von Source-Dateien durch
	\begin{itemize}
		\item Makro-Substitution
		\item bedingte Übersetzung
		\item Einfügen von Dateien
	\end{itemize}
	\item Der Output des Preprocessors wird dem eigentlichen Compiler übergeben
	\item Die Direktiven des Preprocessors beginnen immer mit \#
	\item Der Preprocessor ist zeilenorientiert
\end{itemize}
\vfill\null
\columnbreak
\section{Preprocessor-Direktiven und Bedingungsanweisungen}
\centering
\begin{tabularx}{0.5\textwidth}{|X|X|}
	\hline
	\textbf{Direktiven} & \textbf{Bedingungsanweisungen}\\
	\hline
	\#define & \#if\\
	\hline
	\#undef & \#ifdef\\
	\hline
	\#include & \#ifndef\\
	\hline
	\#line & \#elif\\
	\hline
	\#error & \#else\\
	\hline
	\#pragma & \#endif\\
	\hline
\end{tabularx}
\flushleft
\end{multicols}

\subsection{\#define}
\begin{itemize}
	\item Definition von Makros (Gefahren betrachten!)\hspace{0.05\linewidth}
	\begin{minipage}{0.5\linewidth}
	\vspace{-\baselineskip}
\begin{lstlisting}
#define MAX(a,b) ((a)>(b)?(a):(b))	
\end{lstlisting}
	\end{minipage}
	\item Definition von symbolischen Konstanten (Gefahren beachten!)\hspace{0.05\linewidth}
	\begin{minipage}{0.35\linewidth}
	\vspace{-\baselineskip}
\begin{lstlisting}
#define PI 3.14159265358
\end{lstlisting}
	\end{minipage}
	\item Es erfolgt eine reine Textersetzung ohne Typenprüfung. Diese kann erst der Compiler vornehmen
	\item \emph{\#defines} haben keinen Scope
	\item Simple Definition eines Symbols (z.B. bei Include-Guard)\hspace{0.05\linewidth}
	\begin{minipage}{0.25\linewidth}
	\vspace{-\baselineskip}
\begin{lstlisting}
#define FOO\_H\_
\end{lstlisting}
	\end{minipage}
\end{itemize}

\subsection{\#undef}
Definition eines Symbols rückgängig machen\\
\begin{minipage}{0.2\linewidth}
\vspace{-\baselineskip}
\begin{lstlisting}
#undef MAX
#undef PI
#undef FOO_H_
\end{lstlisting}
\end{minipage}

\subsection{\#include}
Vollständiges Einfügen einer Datei
\begin{minipage}{0.9\linewidth}
\vspace{-\baselineskip}
\begin{lstlisting}
#include <iostream>
// Datei iostream wird in den definierten Include-Verzeichnissen
// gesucht und eingefuegt

#include "foo.h"
// Datei foo.h wird im aktuellen Verzeichnis gesucht und eingefuegt

#include "../drivers/doo.h"
// Pfadangaben muessen immer relativ zum aktuellen Verzeichnis sein.
// Niemals absolute Pfade verwenden!
\end{lstlisting}
\end{minipage}
\vfill
\pagebreak\newpage

\subsection{\#line}
Direktes Setzen der Nummerierung von Sourceode-Zeilen. Durch die optionale Angabe eines Dateinamens lässt sich der Compiler einen neuen Dateinamen unterschieben.\\
Das kann z.B. bei vom Compiler erstellten Dateien nützlich sein.
\begin{minipage}{0.8\linewidth}
\vspace{-\baselineskip}
\begin{lstlisting}
#line 67 "main.cpp"
// die naechste Zeile im Sourcecode erhaelt die Nummer 67
// die Sourcedatei erhaelt den Namen "main.cpp"
\end{lstlisting}
\end{minipage}

\subsection{\#error}
Sofortiger Abbruch des Compiler-Vorgangs und Ausgabe einer Fehlermeldung
\begin{minipage}{0.4\linewidth}
\vspace{-\baselineskip}
\begin{lstlisting}
#ifndef MODEL
#error MODEL ist nicht definiert
#endif
\end{lstlisting}
\end{minipage}

\subsection{\#pragma}
\begin{itemize}
	\item \emph{\#pragma}-Direktiven erlauben die Verwendung von Implementierungs-spezifischen Direktiven. Entwicklungsumgebungen können somit ihre eigenen Anweisungen definieren.\\
	\textbf{Diese Direktive ist damit per Definition nicht portabel.}
	\item Konflikte entstehen keine, da eine Compiler unbekannte \emph{\#pragma}-Direktiven ignoriert.
	\item Die Portabilität ist damit aber nicht gewährleistet. Ein anderer Compiler versteht die Direktive u.U. nicht und erzeugt deshalb nicht den gewünschten Code.
\end{itemize}

\subsection{Bedingungsanweisungen}
\begin{itemize}
	\item Bedingungsanweisungen sind nach folgendem Schema aufgebaut\\
	-----------------\\
	Bedingungsprüfung\\
	Direktiven\\
	beliebig viele \emph{\#elif}-Gruppen mit Direktiven\\
	optional ein \emph{\#else} mit Direktiven
	\emph{\#endif}
	\item Mögliche Bedingungsprüfungen sind
	\begin{itemize}
		\item \emph{\#if} gibt true, falls Bedingung true ist
		\item \emph{\#ifdef} SYMBOL gibt true, falls SYMBOL definiert ist
		\item \emph{\#ifndef} SYMBOL gibt true, falls SYMBOL nicht definiert ist
	\end{itemize}
\end{itemize}
\vfill
\pagebreak\newpage

\subsubsection{Beispiele für Bedingungsanweisungen}
\begin{multicols}{2}
\begin{minipage}{\linewidth}
\vspace{-\baselineskip}
\begin{lstlisting}
#if INT\_MAX > 32767
	int i;
#else
	long i;
#endif

#ifdef TESTVERSION
	printf("Zeilennummer: %d\n", __LINE__);
	printf("Dateiname: %s\n", __FILE__);
#endif

#ifndef CALCULATOR\_H\_
#define CALCULATOR\_H\_
// ...
#endif
\end{lstlisting}
\end{minipage}
\vfill\null
\columnbreak
\begin{minipage}{\linewidth}
\vspace{-\baselineskip}
\begin{lstlisting}
#if 0
	// Auskommentieren des gesamten hier stehenden Codes.
	// Ist sehr nuetzlich waehrend der Entwicklung.
\end{lstlisting}
\end{minipage}
\end{multicols}

\subsection{Weitere Features des Preprocessors}
\begin{itemize}
	\item \#-Operator (stringizing-Operator): Argument wird in String konvertiert
	\item \#\#-Operator (token-past-Operator): Zeichenfolge links und rechts des Operators wird zusammengezogen
	\item und weiteres (siehe Dokus)
	\begin{minipage}{\linewidth}
	\vspace{-\baselineskip}
\begin{lstlisting}
#define SHOW(var, nr) printf(#var #nr " = %.1f\n", var ## nr)

// Annahme: es gibt eine Variable x5 mit dem Wert 16.4
SHOW(x, 5);
// wird umgesetzt in printf("x" "5" " = %.1f\n", x5);
// d.h. in printf("x5 = %.1f\n", x5);

// Ausgabe ist x5 = 16.4
\end{lstlisting}
	\end{minipage}
\end{itemize}
\textbf{na ja!?}

\subsection{Kritische Würdigung des Preprocessors}
\begin{itemize}
	\item Preprocessoranweisungen sollten zurückhaltend eingesetzt werden, da der Code durch zu viele Preprocessoranweisungen sehr schnell unübersichtlich werden kann.
	\item Auf den ersten Blick sind Preprocessoranweisungen oft nicht nachvollziehbar.
	\item Häufig gibt es Alternativen, die ebenso effizient und zudem viel sicherer sind.
	\item Richtig eingesetzt kann der Preprocessor sehr nützlich sein.
\end{itemize}