%!TEX root = ProgCPP_ZF.tex

\part{Vererbung}
\label{sec:Vererbung}

\section{Motivation}
\label{sec:Motivation}
Sie müssen einen Webshop entwickeln. Sie verkaufen Bücher und CDs.\\
Welche Eigenschaften und Methoden benötigen Sie, um ein Buch bzw. eine CD zu charakterisieren?

\section{Artikel als Gemeinsamkeit von Buch und CD}
\label{sec:Artikel als Gemeinsamkeit von Buch und CD}

%%% Bild einfügen %%%

\section{Grundkonzept}
\label{sec:Grundkonzept}
\begin{itemize}
	\item Die Vererbung erlaubt, neue Klassen auf der Basis von bestehenden Klassen zu definieren. Dabei erbt (übernimmt) die neue (Unter-)Klasse alle Eigenschaften der bestehenden (Über-)Klasse.
	\item Man sagt auch, die Oberklasse sei eine Basisklasse, Superclass, Abstraktion oder Generalisierung (Verallgemeinerung).
	\item Die Unterklasse wird auch mit Subclass oder als Spezialisierung bezeichnet.
\end{itemize}

\section{Einsatz der Vererbung}
\label{sec:Einsatz der Vererbung}
\begin{itemize}
	\item Bestehende Klassen erweitern
	\begin{itemize}
		\item zusätzliche Attribute erweitern
		\item zusätzliche Elementfunktionen
	\end{itemize}
	\item Bestehende Methode einer Basisklasse ändern (überschreiben)
	\item Das Finden von guten Basisklassen ist eine Hauptaufgabe in der Designphase.
\end{itemize}

\section{UML-Notation}
\label{sec:UML-Notation}
\begin{itemize}
	\item Generalisierung/Spezialisierung
	\item SubClass erbt sämtliche Eigenschaften von SuperClass
	\item ist-ein Beziehung (SubClass \textbf{ist eine} SuperClass)
	\item Pfeilspitze ist ein geschlossenes Dreieck
\end{itemize}

%%% Bild Folie 8 einfügen %%%

\subsection{"ist ein"-Beziehung}
\label{sec:"ist ein"-Beziehung}
"ist ein"-Beziehung ("is a"-relationship)\\
Beispiel: Baum \textbf{ist eine} Pflanze, Blume \textbf{ist eine} Pflanze.

\section{Beispiel: Vererbungshierarchie Lebewesen}
\label{sec:Beispiel: Vererbungshierarchie Lebewesen}

%%% Bild Folie 10 einfügen %%%

\subsection{C++-Syntax}
\label{sec:C++-Syntax}
\noindent
\begin{minipage}{\linewidth}
	\begin{lstlisting}
	class SubClass *@\color{red}: public SuperClass\color{black}@*
	{
		public:
		
		protected:
		
		private:
		
	};
	\end{lstlisting}
\end{minipage}
public ist Normalfall (private und protected sind auch möglich).

\subsection{Zugriff auf Elemente der Basisklasse}
\label{sec:Zugriff auf Elemente der Basisklasse}

%%% Bild Folie 12 %%%

\section{Spezifikation von Basisklassen}
\label{sec:Spezifikation von Basisklassen}
\begin{itemize}
	\item Grundsatz: Vererbung sollte immer public sein (zu 99,99\%)
	\item Falls bei Vererbung protected oder private in Betracht gezogen wird, kann der Grund dafür eine falsche Verwendung der Vererbung sein
	\item Für ganz spezifische Anwendungen kann die Vererbung mit protected oder private sinnvoll sein
\end{itemize}

\section{Beispiel: ComicCharacter (Comics01)}
\label{sec:Beispiel: ComicCharacter (Comics01)}

%%% Comics01 einfügen
%%% Bild (Klassen) von Folie 14


\section{Einsatz von protected bei Klassenelementen}
\label{sec:Einsatz von protected bei Klassenelementen}
\begin{itemize}
	\item Bei Datenelementen (Attributen) soll protected grundsätzlich nicht eingesetzt werden. Attribute sollen generell private sein.
	\item Bei Elementfunktionen kann es in Einzelfällen sinnvoll sein, diese als protected zu definieren. Dadurch wird der Zugriff gegenüber einer public-Sichtbarkeit auf die abgeleiteten Klassen beschränkt.
\end{itemize}

\section{Objektgrösse bei der Vererbung}
\label{sec:Objektgrösse bei der Vererbung}
\begin{itemize}
	\item Ein Objekt einer vererbten Klasse enthält alle Teile der Basisklasse(n) und zusätzlich noch die spezifischen eigenen Teile.
	\item Das Objekt ist somit mindestens so gross wie jenes der Basisklasse(n). (es gibt keine Vererbung "by reference")
	\item Wenn Vererbung schlecht eingesetzt wird (z.B. keine is-a-Beziehung), können unnötig grosse Objekte entstehen.
	\item Bei einer Aggregationsbeziehung kann durchaus eine Referenz (oder ein Pointer) auf ein anderes Objekt verwendet werden, d.h. Aggregation "by reference" ist möglich.
\end{itemize}

\section{Schlechter (falscher) Einsatz von Vererbung}
\label{sec:Schlechter (falscher) Einsatz von Vererbung}
Zwischen Wald und Pflanze besteht nicht eine "ist-ein" Beziehung. Die richtige Beziehung wäre "hat-ein", da ein Wald mehrere Pflanzen hat (folgt später).

%%% Bild Folie 17

\section{Substitutionsprinzip}
\label{sec:Substitutionsprinzip}
\begin{itemize}
	\item Ein Objekt einer Oberklasse kann Objekte einer beliebigen Unterklasse aufnehmen.
	\item Ein Objekt einer Unterklasse kann keine Objekte der Oberklasse aufnehmen.
	\item[\-]
	\noindent
	\begin{minipage}{\linewidth}
		\begin{lstlisting}
		class SuperClass{}; 
		
		class SubClass : public SuperClass {};
		
		SuperClass super;
		SubClass sub;
		super = sup; 	*@\color{green}// ok\color{black}@"
		sub = super;	*@\color{red}// geht nicht \color{black}@*
		\end{lstlisting}
	\end{minipage}
\end{itemize}












