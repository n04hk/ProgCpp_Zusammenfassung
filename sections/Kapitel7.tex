%!TEX root = main.tex

\part{Eclipse IDE}
\label{sec:Eclipse IDE}

\section{Eclipse}
\label{sec:Eclipse}
\begin{itemize}
	\item Integrated Development Environment
	\item[\-] (IDE, Integrierte Entwicklungsumgebung)
	\item Open-Source Software (OSS)
	\item Offene, erweiterbare Architektur basierend auf Plug-Ins
	\item Implementiert in Java
	\item Auf verschiedenen Plattformen lauffähig (multi-platform)
	\item Für unterschiedliche Programmiersprachen (multi-language)
\end{itemize}

\subsection{Workspace}
\label{sec:Workspace}
\begin{itemize}
	\item Der Workspace enthält vom Benutzer definierte Daten (Projekte und Ressourcen wie Ordner und Files)
	\item Er enthält alle User-Metadaten (Code, Scripts, Database objects, Konfigurationsdaten)
	\item Ein Benutzer arbeitet zu einer bestimmten Zeit genau in einem einzigen Workspace
\end{itemize}

\subsection{Ressourcen (Resources)}
\label{sec:Ressourcen (Resources)}
\begin{itemize}
	\item Oberbegriff für
	\begin{itemize}
		\item Projekte
		\item Ordner
		\item Files
	\end{itemize}
	\item Üblicherweise in einer hierarchischen Struktur betrachtet
	\item Können editiert werden
\end{itemize}

\subsection{Project}
\label{sec:Project}
\begin{itemize}
	\item Ein logisches Speicherkonzept für die Speicherung von Programmen
	\item Gehört einem Workspace an
	\item Ist implementiert als Verzeichnis in einem Workspace
\end{itemize}

\subsection{Debugger}
\label{sec:Debugger}

\subsubsection{Testen und Debugging}
\label{sec:Testen und Debugging}
\begin{itemize}
	\item Testen und Debugging sind zwei unterschiedliche Prozesse
	\item Das Ziel eines Tests ist, Fehler zu finden
	\item Das Ziel des Debuggings ist, diese Fehler zu lokalisieren und zu korrigieren
\end{itemize}

\subsubsection{Funktionen eines Debuggers}
\label{Fukntionen eines Debuggers}
\begin{itemize}
	\item Betrachten von Variablenwerten
	\item Unterbrechung des Programmablaufs mit Breakpoints
	\item Schrittweise Ausführung von Programmen (Step into, Step out)
\end{itemize}

\subsubsection{Assertions (Zusicherungen)}
\label{sec:Assertions (Zusicherungen)}
\noindent
\begin{minipage}{\linewidth}
\begin{lstlisting}
#include <cassert>
assert(i>0);

// in C:
#include <assert.h>
assert(i>0);
\end{lstlisting}
\end{minipage}
\begin{itemize}
	\item Zweck:
	\item[\-] Überprüfung von logischen Annahmen während der Entwicklungsphase, speziell für die Überprüfung von Anfangs- und Endbedingungen in einer Funktion
	\item Das Programm bricht mit einer Fehlermeldung ab, falls das Argument den bool'schen Wert false besitzt. Im Beispiel oben: Abbrich, falls i <= 0
\end{itemize}

\subsubsection{Zu beachten bei Assertions}
\label{sec:Zu beachten bei Assertions}
\begin{itemize}
	\item assert() ist wirkungslos, wenn ohne Debugschalter compiliert wird. Dies ist in der (Release-)Version der Fall, die ausgeliefert wird
	\item Bei den assert()-Anweisungen darf deshalb kein Nebeneffekt programmiert werden, da dieser in der ausgelieferten Version fehlen würde
	\item Beispiel:
	\item[\-] assert(openFile(filename) == ok);
	\item[\-] Wenn ohne Debugschalter compiliert wird, würde das File nicht mehr geöffnet
\end{itemize}